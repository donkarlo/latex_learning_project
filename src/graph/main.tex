\documentclass{book}

% Graphics
\usepackage{tikz}
%% fit is for process I think
\usetikzlibrary{graphs,graphdrawing,arrows.meta,mindmap,fit,positioning}
\usegdlibrary{layered,force}

% For hierarchical trees
\usepackage[edges]{forest}

% Enforce non-floating placement with [H]
\usepackage{float}
\usepackage{caption}
% moves all figure captions
\captionsetup{singlelinecheck=false}

\begin{document}

\section{Layered}
    For vertical or horizental expansion
    \begin{figure}[H]
        \begin{tikzpicture}[>=Stealth, nodes={draw,circle,inner sep=2pt}]
            \graph [layered layout, level distance=16mm, sibling distance=10mm] {
                A -> {B, D, E};
                B -> C;
                D -> {C, E, F};
                C -> F;
                E -> F;
            };
        \end{tikzpicture}
        \caption{Layered graph}
        \label{fig:layered-graph}
    \end{figure}
    
    \paragraph{With ellipses and curves}
        With ellipses and some bending
        \begin{figure}[H]
            \begin{tikzpicture}[>=Stealth,
                nodes={draw,rectangle,align=center,inner sep=3pt}]
                \graph [layered layout, level distance=16mm, edges={bend left=10},
                    grow down] {
                    A -> {Bfdsdfs, Dsdfs, E};
                    A -> B;
                    A -> D;
                    B -> C;
                    D -> {Csdf, Edsfs fdfsd, F};
                    C -> Fdfs sgfsd;
                    E -> F;
                };
            \end{tikzpicture}
            \caption{Layered graph with horizontal ellipses}
            \label{fig:layered-graph-with ellipse}
        \end{figure}
        
\section{Spring}
    For free figures based on force
    \begin{figure}[H]
        \raggedright
        \begin{tikzpicture}[>=Stealth, nodes={draw,ellipse,inner sep=5pt}]
            \graph [spring layout] {
                Ax -> {B, D, E};
                B -> C;
                D -> {C, Ex, F};
                Cc -> Fc;
                E -> Fcc ;
            };
        \end{tikzpicture}
        \caption{Spring layout graph}
        \label{fig:spring-graph}
    \end{figure}
    
\section{Mindmap}
    \begin{figure}[H]
        \raggedright
        \begin{tikzpicture}
            \path[mindmap,concept color=black!40,text=white]
            node[concept] {Self-awareness}
            [clockwise from=0, level 1 concept/.append style={concept color=blue!60}]
            child[concept color=blue!60] { node[concept] {Sensing}
            child { node[concept] {Proprio} }
            child { node[concept] {Extero} } }
            child[concept color=green!60] { node[concept] {Models}
            child { node[concept] {DBN} }
            child { node[concept] {Coupled} } }
            child[concept color=orange!70] { node[concept] {Novelty}
            child { node[concept] {KL} }
            child { node[concept] {Hausdorff} } };
        \end{tikzpicture}
        \caption{Mindmap of self-awareness}
        \label{fig:mindmap}
    \end{figure}
    
\section{Tree and forest}
    \begin{figure}[H]
        \raggedright
        \begin{forest}
            for tree={draw, rounded corners, align=center, s sep=8mm, l sep=6mm}
            [Self-awareness
            [Sensing
            [Proprio]
            [Extero]
            ]
            [Models
            [DBN]
            [Coupled DBN]
            ]
            [Novelty
            [KL]
            [Hausdorff]
            ]
            ]
        \end{forest}
        \caption{Forest tree of self-awareness}
        \label{fig:forest-tree}
    \end{figure}
    
\section{Process}
    \begin{tikzpicture}[
        box/.style={draw, rounded corners, align=center, inner sep=6pt},
        >=Stealth]
        \node[box] (sense) {Sensors\\(LiDAR, GPS)};
        \node[box, right=18mm of sense] (features) {Preprocess\\ + Derivatives};
        \node[box, right=18mm of features] (model) {Coupled DBN};
        \node[box, below=12mm of model] (novel) {Novelty Score\\(KL, Hausdorff)};
        \draw[->] (sense) -- (features);
        \draw[->] (features) -- (model);
        \draw[->] (model) -- (novel);
    \end{tikzpicture}

\end{document}
